\chapter{Algorithm: Off-policy deterministic AC}
\label{sec:algo}
\section{Distributional Deterministic Policy Gradients}

We introduce an off-policy actor-critic distributional algorithm.
The actor uses a distributional variant of the deterministic policy gradient algorithm.

\subsection{Off-policy Deterministic policy gradient algorithm}
We will use an actor-critic approach based on the DPG algorithm \citep{silver2014b}
which uses deterministic policies $a=\mu_{\theta}(s)$.
From a practical viewpoint, using stochastic policies requires integrating over both
state and action spaces to compute the policy gradient, whereas the deterministic case
only needs to integrate over the state space. Hence, stochastic policy gradients
may require much more samples, especially if the action space has many dimensions.

In general, behaving according to a deterministic policy does not ensure adequate exploration,
and may lead to suboptimal solutions. 
However, if the policy is deterministic, the expected cumulative reward in the next-state
depends only on the environment. This means that it is possible to learn the value function
Q under policy $\mu$ off-policy, ie using transitions which are generated from a different stochastic
behavior policy $\beta$ which act on the environment and ensures enough exploration.
that ensures enough exploration. will use an off-policy actor-critic.
An advantage of off-policy algorithms is that we can treat the problem of exploration
independently from the learning algorithm.

Q-learning \cite{Watkins1992}, a commonly used off-policy algorithm, uses the greedy policy
$\mu(s)=\underset{a}{\text{argmax}} Q(x,a)$.
In  a  continuous  action  space, it is not possible to apply Q-learning straight-forward because finding the 
greedy policy requires an optimization of a at every timestep, which is too slow to be practical with large action spaces.
In this case, actor-critic methods are commonly used, where action selection is performed through a
separate policy network, known as the actor,and updated with respect to a value estimate,
known as the critic \cite{Sutton1998}.
The policy can be updated following the deterministic policy gradient theorem \cite{silver2014b},
which corresponds to learning an approximation to the maximum of Q,
by propagating the gradient through both policy and Q.
Specifically maximizing, the performance objective which is the value function of the target policy $\mu$, averaged over the state
distribution of the behavior policy $\beta$:

\begin{align}
    J_\beta(\mu | \theta^\mu) = \int_\mathcal{X} \rho^\beta(s)Q^{\mu}(x,\mu (x| \theta^\mu))dx \nonumber\\
    \nabla_{\theta^\mu} J_\beta(\mu | \theta^\mu) \approx \mathbb E_{x \sim \rho^\beta} \label[]{eq:off_policy_gradient}
    \big [\nabla_{\theta^\mu} \mu(x,| \theta^\mu) \nabla_a Q^{\mu}(x,a)|_{a=\mu(x| \theta^\mu)}  \big]
\end{align} 


\eqref{eq:off_policy_gradient} gives the off-policy deterministic policy gradient, which was proved
by \cite{silver2014b}  to be the policy gradient, ie the gradient of the policy's performance.
A term that depends on $\nabla_\theta Q^{\mu_\theta}(x,a) $ has been dropped in \, following a justification
given by \cite{Degris2012} that argues that this is a good approximation since it can
preserve the set of local optima to which gradient ascent converges.

Similarly to \cite{Lillicrap2016}, we will use neural networks as non-linear function approximators
for learning both action-value functions and the deterministic target policy.

\subsection{Distributional approach}
\subsubsection{Critic}
Instead of using the standard critic network that approximates the value function,
we will use the distribution variant which maps from state-action pairs to distributions, similar
to the implicit quantile network (IQN) introduced in \cite{Dabney2018a}.\\
IQN is a deterministic parametric function trained to reparameterize samples from a
base distribution, e.g $\tau \in U([0,1])$, to the respective
quantile values of a target distribution.
We define $F^{-1}_Z(\tau):=Z(x,a;\tau)$ as the quantile function at $\tau \in [0,1]$ for the random variable $Z(x,a)$.
Thus, for $\tau \in U([0,1])$, the resulting state-action return distribution sample is
$Z(x,a;\tau)\sim Z(x,a)$

The parameters $\theta^Z$ of the IQN network are updated using backpropagation of the sampled 
quantile regression loss.
The quantile regression loss is computed on the sampled temporal-difference error
using the distributional Bellman operator:
For two samples $\tau, \tau' \sim U([0,1])$, and current policy $\mu_{\theta}$, the sampled
TD error is:

\begin{align}
    \delta^{\tau, \tau'} = r + \gamma Z(x',\mu(x');\tau'| \theta^\mu)-Z(x,a;\tau)
\end{align}
Then, we compute the quantile regression loss:
\begin{equation}
    \mathcal{L}(x,a,r,x')= \frac{1}{N}\sum_{i=1}^{N}\sum_{j=1}^{N'}\rho_{\tau_i}(\delta^{\tau_i, \tau_j'})
\end{equation}

where $\rho$ is as defined in \eqref{eq:quantile_loss} and 
where N and N' are the number of iid samples $\tau_i, \tau_j' \sim U([0,1]$ used to estimate the loss.

\subsubsection{Actor}
The policy is updated via deterministic policy gradient ascent.
We modify equation  \eqref{eq:off_policy_gradient}, to include the action-value distribution.

\begin{align}
    \nabla_{\theta^\mu} J_\beta(\mu | \theta^\mu) &\approx \mathbb E_{x \sim \rho^\beta} 
    \big [\nabla_{\theta^\mu} \mu(x,| \theta^\mu) \nabla_a Q^{\mu}(x,a)|_{a=\mu(x| \theta^\mu)}  \big]\label{eq:distr_dpg_alg1}\\
    &=\mathbb E_{x \sim \rho^\beta} 
    \big [\nabla_{\theta^\mu} \mu(x,| \theta^\mu) \mathbb E [\nabla_a Z(x,a | \theta^Z)]|_{a=\mu(x| \theta^\mu)}  \big]
    \label{eq:distr_dpg_alg2}
\end{align}



\section{Risk}

The step from \eqref{eq:distr_dpg_alg1} to \eqref{eq:distr_dpg_alg2} $\mathbb E [\nabla_a Z(x,a | \theta^Z)]$ comes by the fact that
\begin{equation}
    Q(x,a) = \mathbb E[Z(x,a)] \label{eq:neutral_policy}
\end{equation}
We could make use of the information provided by the distribution over returns to learn 
risk-sensitive policies, ie do not maximize the expected valued of the cumulative reward but other metrics
that take into account risk of the actions.
To approach this, either we use a performance objective that aims to maximize a concave utility function that
gives rise to a risk-averse policy:

\begin{equation}
    \pi(x)= \underset{a}{\text{argmax}} \mathbb E_{Z(x,a)} [U(z)]
\end{equation}

or equivalently, we distort the cumulative probabilities of the random variable Z
using a \textit{distortion risk measure} and compute the expectation of the reweighted
distribution under this distortion measure.

We propose to use the later approach, ie use as a performance objective 
the distorted expectation of  Z(x,a) under the distortion risk measure $\beta: [0,1] \ra [0,1]$,
which transforms equation  \eqref{eq:neutral_policy} can be converted to a more general one:
\begin{equation}
    Q_\beta(x,a) = \mathbb E_{\tau \sim U([0,1])} [Z_{\beta(\tau)}(x,a)] \label{eq:risk_policy}
\end{equation}

Since distorted expectations can be expressed as weighted average over the quantiles \cite{Dhaene2012},
we can use a specific sampling base distribution $\beta: [0,1] \ra [0,1]$
to sample the quantile levels $\tau \in[0,1])$ from our critic network $Z(x,a; \tau)$.

As stated, our goal is to maximize the CVaR:
\begin{equation}
    \underset{\mu} \max \text{CVaR}_\alpha [Z (x_0, \mu(x_0))] 
\end{equation}

We remind ourselves of its definition as a weighted sum over the quantiles under the distortion risk measure:
$\beta(\tau) = \alpha \tau$ as :
\begin{align}
\text{CVaR}_\alpha (Z) &= \int_{0}^1 F^{-1}_Z(\tau)d\beta(\tau)
\end{align}

Hence:
\begin{align}
\text{CVaR}_\alpha (Z)&=\frac{1}{\alpha} \int_{0}^{\alpha} F^{-1}_Z(\tau) d\tau\label{eq:cvar_1}
\end{align}

We can hence approximate \eqref{eq:cvar_1} via sampling, by taking k samples of 
$\hat\tau :=\beta(\tau) \text{ where } \tau \sim U[0,1]$, i.e $\hat\tau \sim U[0,\alpha]$:
\begin{equation}
\text{CVaR}_\alpha (Z) \approx \frac{1}{\alpha} \frac{1}{k}\sum_{i=1}^kZ(x,a; \hat\tau_i)
\qquad \hat\tau_i \sim U[0,\alpha] \quad \forall i \in [1,k]
\end{equation}

Finally, then, we use the following distributional policy gradient for the actor network:

\begin{align}
    \nabla_{\theta^\mu} J_\beta(\mu | \theta^\mu) &\approx \mathbb E_{x \sim \rho^\beta} 
    \big [\nabla_{\theta^\mu} \mu(x,| \theta^\mu) \nabla_a  [\frac{1}{\alpha} \frac{1}{k}
    \sum_{i=1}^kZ(x,a; \hat\tau_i) | \theta^Z)]|_{a=\mu(x| \theta^\mu)}  \big]
    \label{eq:actor_grad}
\end{align}

It is important to notice the capability of the algorithm to be implemented off-policy.
Stochastic off-policy actor-critic algorithms, as presented in \citet{Degris2012}, typically
use importance sampling for both actor and critic. 
However, when using deterministic policies the policy gradient doesn't include the integral over actions,
and hence we can avoid importance sampling in the actor.
Additionally, by using "Q-learning for continuous action spaces", we can avoid importance sampling in the critic.


\section{Technical Details of the algorithm}


\begin{itemize}
    \item Explain replay buffer characteristics
    \item Target networks and update frequencies
    \item Adam optimizers (p4 lillicrap)
	\item Ornstein Noises and exponential decays
	\item Add pseudo-code of the algorithm 
\end{itemize}














