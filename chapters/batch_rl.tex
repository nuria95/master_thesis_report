\chapter{Batch RL}
\begin{itemize}
    \item Watch videos Levine
    \item Papers bear and bcq
    
\end{itemize}

We decide to test the capabilities of our algorithm in a \textit{fully} off-policy setting, 
also called \textit{batch RL setting} or \textit{offline} RL. In this setting, the agent 
can only learn from a fixed dataset without further interaction with the environment.

\todo{ Add motivation in doing so: envs where collection of data is expensive, unsafe for robotics or autonomous vehicles}

The 'off-policy' algorithm we presented in previous sections falls in the category of 
off-policy \textit{"growing batch learning"} in which data is collected by using near-on-policy
policies such as $\epsilon$-greedy and stored in a replay buffer.
After used for training, the data is replaced with \textit{fresher} data obtained from
interaction of the agent with the environment using an updated policy.
As a result, the dataset used tends to be heavily correlated to the current policy.


\textbf{Issues with Batch RL}

Most of off-policy algorithms fail to learn in the off-line setting. 
This is due to a fundamental problem of off-policy RL, called extrapolation error \citep{Fujimoto2019} or 
bootstrapping error \citep{Kumar2019}. This error is introduced due to a mismatch
between the dataset distribution and the state-action visitation distribution induced by the
current target policy.
At every train step the Q estimate is updated in the direction to reduce
the Bellman error, ie the mean squared error between the current value estimate and
the expected Q value under the current target policy at the next state.
The Q function estimator, however, is valid only when evaluated on actions sampled from the behavior policy,
which in the batch-RL case is the distribution of the dataset.
Using unfamiliar (unlikely or not contained in the dataset) action
(also called out of distribution (OOD) actions in \citep{Kumar2019}) for the next-state,
results on a new Q value estimate which is affected by this extrapolation error,
resulting in pathological values that incur large absolute error from the optimal desired Q-value.

It is good to notice, that for an on-policy settings, extrapolation error is generally something positive, since it
leads to a beneficial exploration. In this case, if the value function is overestimating the value at
a (state-action) pair, the current policy will lead the agent to that pair, collect the data at that point
and hence, the value estimate will be corrected afterwards.
In the off-policy setting, the correction step is not possible due to the inability of collecting new data.



\textbf{Our approach}
\todo{rewrite}
To overcome this issue, we inspire ourselves on the approach presented in \citet{Fujimoto2019}, where a
generative model $G_w$ is trained to generate actions with high similarity to the dataset.
For the generative model we use a conditional variational auto-encoder (VAE) \cite{Kingma2014} which
generates action samples as a reasonable approximation to
$\underset{a}{\text{argmax}}  P _\mathcal{B}^G(a|s)$, where $P _\mathcal{B}^G(a|s)$ is the 
conditioned marginal likelihood.

\section{Details of VAE}

A variational autoencoder aims to maximize the marginal log-likelihood $\log p(X) = \sum_{i=1}^{N}\log p(x_i)$
, where X is the dataset with iid samples $\{x_{1},x_2,...x_N\}$.
It is assumed that data is generated by some random process, involving an unobserved continuous
random variable \textbf{z}.
The process consists of two steps: (1) a value \textbf{$z_i$} is generated from some prior distribution p(z) and (2)
a value $x_i$ is generated from some conditional distribution $p(x|z)$.
Given that the true probabilities are unknown, a recognition model $q(z|x; \phi )$ is introduced
as an approximation to the intractable true posterior $p(z|x; \theta)$.

The recognition model $q(z|x; \phi)$ is called an \textit{encoder},  since given a datapoint x it produces
a \textit{random latent vector} z.
$p(x|z; \theta)$ is called a \textit{decoder},
since given the random latent vector z it reconstructs the original sample x.

Since computing the desired marginal $p(X; \theta)$ is intractable, VAE algorithm optimizes a lower bound instead:

\begin{equation}
    \log p(X; \theta) \geq \mathcal{L}(\theta, \phi; X) = \mathbb E_{q(z|X;\phi)} [\log p(X|z; \theta)] - D_{KL}(q(z|X;\phi)\; ||\;p(z; \theta))
\end{equation}

For our implementation, the prior $p(z; \theta)$ is chosen to be a multivariate normal
distribution $\mathcal{N}\big( 0,Id\big )$, hence it lacks parameters.

For the probabilistic encoders and decoders we use neural networks.
For the encoder $q(z|x_i;\phi)$ we used a neural network with  Gaussian output, specifically a 
multivariate Gaussian with a diagonal covariance structure $\mathcal{N}(z | \mu(X), \sigma^2(X)Id)$,
where $\mu$ and $\sigma$ are the outputs of the neural network, i.e nonlinear 
functions of datapoint $x_i:=(state_i,action_i)$ and $\phi$.
To sample from the posterior $z_i \sim q(z|x_i; \phi)$ we use the reparameterization trick:
$z_i = g(x_i, \epsilon; \phi)=\mu_i + \sigma_i \odot \epsilon$ where $\epsilon \sim  \mathcal{N}(0,Id)$
and $\odot$ is the element-wise product.
For the decoder $p(x|z; \theta)$ we used another neural network with deterministic output, ie nonlinear function of 
datapoint $\hat x_i:=(state_i,z_i)$ and $\theta$.


The VAE is trained to maximize  reconstruction loss and a KL-divergence term according to the distribution of
the latent vector:

When it comes to training the VAE, both recognition model parameters $\phi$ and the generative model parameters $\theta$ are
learnt jointly to maximize the variational lower bound $\mathcal{L}(\theta, \phi; X)$ via gradient ascent which includes the
expected reconstruction error loss and the KL-divergence term according to the distribution of the latent vectors.
When both prior and posterior are Gaussian, KL-divergence can be computed analytically:
\begin{equation}
    - D_{KL}(q(z|X;\phi)\; ||\;p(z; \theta)) = \frac{1}{2}\sum_{j=1}^J (1+\log((\sigma_j)^2)-(\mu_j)^2-(\sigma_j)^2)
\end{equation}

where J is the dimensionality of z, and $\mu_j,\sigma_j$ represent the jth element of these vectors.
The expected reconstruction error 
$\mathbb E_{q(z|X;\phi)} [\log p(X|z; \theta)]$
requires estimation by sampling, and we will use the mean-squared error between the $action_i$ from the dataset and the reconstructed
action.

Finally, when acting during evaluation or deployment, random values of z will be sampled from the multivariate normal
and passed through the decoder to produce actions.



Change the Deterministic actor network to:

VAE(state) + perturbation\_level $\times$ DeterministicActorNN(vae\_action, state)

Train DeterministicActorNN as in the algorithm presented previously for off-policy RL.