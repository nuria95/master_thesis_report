\chapter{Conditional Value at Risk (CVaR)}
\label{sec:cvar}
\section{CVaR}

We focus on the importance of \textit{value distribution}, the distribution of the random return received by a RL
agent, in contrast to the common approach in RL of modelling the expectation of this return.
The latter neither takes int account the variability of the cost (i.e fluctuations around the mean), nor its sensitivity to modeling errors. \cite{Chow2015}

We aim to learn this distribution and try to minimize other metrics rather than its mean, which can be crucial for some environments 
in which ensuring that the cost is always above a certain value with certain probability is crucial.

A metric that has recently gained a lot of popularity is the Conditional Value at Risk, eg in finance, due to its favorable computation properties 
and superior ability to safeguard a decision maker from the "outcomes that hurt the most" \cite{Serraino2013}

\subsection{Conditional Value-at-Risk (CVaR)}

Let Z be a bounded-mean random variable, i.e $\mathbb E[|Z|] < \infty$, on a probability space $(\Omega, \mathbb F, \mathbb P)$, with cumulative distribution
function $F(z) = \mathbb P (Z \leq z)$. We interpret Z as a reward.
The value-at-risk (VaR) at confidence level $\alpha \in (0,1) $ is the $\alpha$ quantile of Z, i.e, 
$\text{VaR}_\alpha (Z) = \inf \{ z \hspace{1mm} | \hspace{1mm}F(z) \geq  \alpha  \}$.
The conditional value-at-risk (CVaR) at confidence level $\alpha \in (0,1) $ is defined as
the expected reward of outcomes worse than the $\alpha$-quantile ($\text{VaR}_\alpha$):
\begin{equation}
    \text{CVaR}_\alpha (Z) = \frac{1}{\alpha} \int_{0}^{\alpha} F^{-1}_Z(\beta) d\beta=\frac{1}{\alpha} \int_{0}^{\alpha} \text{VaR}_\beta (Z) d\beta
 \end{equation}

 Rockafellar and Uryasev \cite{Rockafellar2000} also showed that CVaR is equivalent to the solution of
 the following optimization problem:

\begin{equation}
    \text{CVaR}_\alpha (Z) = \underset{\nu} \max \big\{\nu + \frac{1}{\alpha} \mathbb E_Z[[Z- \nu]^-]\big\}
\end{equation}

where $(x)^- = \min(x,0)$. In the optimal point it holds that $\nu^*=\text{VaR}_\alpha(Z)$.

A useful property of CVaR, is its alternative dual representation \cite{Artzner1999}:

\begin{equation}
    \text{CVaR}_\alpha (Z) = \underset{\xi \in U_{\text{CVaR}} (\alpha, \mathbb{P})} \min \mathbb E_\xi[Z]
\end{equation}

where $\mathbb E_\xi[Z]$ denotes the $\xi$-weighted expectation of Z, and the risk envelope $U_\text{CVaR}$ is
given by:

\begin{equation}
    U_{\text{CVaR}}(\alpha, \mathbb{P}) = \Big\{\xi | \xi(w)  \in \big [ 0, \frac{1}{\alpha} \int_{w\in\Omega} \xi(w)\mathbb{P}(w)dw=1   \big ] \Big\}
\end{equation}

Thus, the CVaR of a random variable may be interpreted as the worst case expectation of Z, under
a perturbed distribution $\xi \mathbb{P}$.
