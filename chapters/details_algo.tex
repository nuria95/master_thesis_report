\section{Technical Details of the algorithm} \label{sec:technical_details}
The algorithm uses neural networks as non-linear function approximators for both the actor and the critic.
To make effective use of large neural networks, we use insights fom the DeepDPG algorithm \cite{Lillicrap2016} and in its turn
from Deep Q Network (DQN) \citep{Mnih2015}.
Before DQN, RL was believed to be unstable or diverge when nonlinear function approximators such as neural networks were used.
The instability has several causes. First, correlation in the sequence of observations occurring when
samples ar generated from exploring sequentially in an environment (hence i.i.d assumption is violated).
Second, non-stationary data distribution since small updates in Q network may significantly change the policy and hereby, the data distribution.
And third, correlation between the action-values and the target values during temporal difference backups since same network is being used.
These 3 issues are solved by adding 2 innovations: 1)  \textit{experience replay buffer}: the network is trained by sampling 
batches of random past observations from a buffer, hereby removing correlations in the observation sequence and smoothing over
changes in the data distribution.
2) use \textit{target Q networks} that are only periodically updated, to update the Q network during temporal difference backups to reduce
correlations with the target.

Observations $(x_t,a_t,r_t,x_{t+1})$ were sampled from the environment when acting under behavior policy $\beta$ and stored
in a fixed-sized (we used size of $10^6$). When full, oldest samples were discarded.
Since the algorithm is off-policy, the buffer can be large, allowing the algorithm to benefit from
learning across a set of uncorrelated transitions.
Exploration was addressed by constructing an exploratory behavior policy $\beta$ by adding noise $\eta$ to the actor policy.

\begin{equation}
    \beta(x_t)=\pi(x_t| \theta^\pi) + \eta
\end{equation}
where $\eta$ is sampled from a noise process. In our case, as in \citet{Lillicrap2016},we
used an Ornstein-Uhlenbeck process \citep{Uhlenbeck1930} with $\theta = 0.15$ and $\sigma=0.3$ with exponential decay.
Ornstein-Uhlenbeck process models the velocity of a Brownian particle with friction, which results in temporally correlated
values centered around the mean ($\mu$=0.) \todo{explain more?}

At every training step, a minibatch of observations is sampled from the buffer.
To estimate the target value for critic training, we use two target networks:  \textit{critic target network} and \textit{actor target network}
($\hat{Z}(x,a;\tau | \theta^{\hat{Z}})$ and  $\hat{\pi}(x | \theta^{\hat{\pi}})$) which are initialized
as their homologues \todo{rename?}, but constrained to slowly track the learnt networks, so that to stabilize learning.
With this goal, weights of target networks are updated ``softly", via:
$\theta' \leftarrow (1-\tau)\theta' + \tau\theta \; \text{where} \; \tau \in [0,1], \tau \lll 1$.

We additionally, use two insights from TD3 \citet{Fujimoto2018}.
First, we update the policy and target networks less frequently than the critic network. As 
\citet{Fujimoto2018} recommends we do one policy and target networks updates for every 
two critic network updates. 

Second, we use target policy smoothing to address a particular failure mode that can happen
in DeepDPG. To prevent policy to exploit 
actions for which the critic overestimated its value, target policy smoothing adds noise to target actions
to smooth out Q over similar actions. Specifically target action used for the TD backup:
\begin{equation}
    a_{t+1}=\hat{\pi}(x_{t+1} | \theta^{\hat{\pi}} + \text{clip}(\epsilon, -c, c) \quad \epsilon \sim \mathcal{N}(0,\sigma)
\end{equation}
where we experimentally set $c=0.5$ and $\sigma=0.2$ and $\mathcal{N}(0,\sigma)$ is a
Gaussian distribution with mean=0 and standard deviation $\sigma$.


To compute the quantile regression loss, we used N'=N=32 quantile levels to sample from
target and critic networks.This parameter must be kept relatively big but further increasing
it doesn't seem to improve performance.
To compute the empirical CVaR from the estimated value distribution 
$Z(x,a; \tau) | \theta^Z)$, we use K=8 quantile levels.


Training starts after a warm-up of T=25000 environment interactions. During this time, actor
networks where not used but random policies where used instead, to enhance exploration.
Adam optimizer wsa used for learning the neural networks parameters with a learning rate of $10^-4$ and $10^-3$
for the actor and the critic respectively.
For additional information on the networks used, please see Appendix \ref{chap:appendix}





\begin{algorithm}[H]
\SetAlgoLined
\KwResult{Write here the result }
 initialization\;
 \While{While condition}{
  instructions\;
  \eIf{condition}{
   instructions1\;
   instructions2\;
   }{
   instructions3\;
  }
 }
 \caption{How to write algorithms}
\end{algorithm}





From a practical viewpoint, using stochastic policies requires integrating over both
state and action spaces to compute the policy gradient, whereas the deterministic case
only needs to integrate over the state space. Hence, stochastic policy gradients
may require much more samples, especially if the action space has many dimensions.



use replay buffer, target networks as DQN to deal with problems of Q-learning with functions approximators.

Since distorted expectations can be expressed as weighted average over the quantiles \cite{Dhaene2012},
we can use a specific sampling base distribution $\beta: [0,1] \ra [0,1]$
to sample the quantile levels $\tau \in[0,1])$ from our critic network $Z(x,a; \tau)$.