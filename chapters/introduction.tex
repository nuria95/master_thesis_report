\chapter{Introduction}
\label{sec:introduction}


In sequential decision-making problems, an agent interacts with an environment by
selecting actions according to a policy 
and in turn, it observes the state transitions of the system and it receives a reward.\\
During this interaction different sources of irreducible 
randomness are introduced in the system due to modelling errors or inherent stochasticity of the environment.
% The decision-maker or \textit{agent} uses a performance measure to assess how good actions were.\\
The most common optimization criterion for the decision maker to assess how good actions were, 
is the expectation of the cumulative reward
with respect to the randomness in the system. This leads to the so called \textit{risk-neutral} behavior.

Notion of risk is related to the fact that even an optimal policy with respect to the 
expected return may perform poorly in some cases.
Since maximizing the expected return does not necessarily imply the avoidance
of rare occurrences of large negative outcomes,
we need other criteria to evaluate risk. \citep{Garcia2015}.

Risk-sensitive decision-making provides a promising
approach to compute robust and safe policies, essential in safety-critical applications,
such as autonomous navigation.
However, finding computationally tractable and conceptually meaningful methodologies for
such a goal is non-trivial and still a challenge.

In this thesis, we will work on the reinforcement learning (RL)
framework in which a model of the environment is unknown and the agent
must discover which actions yield the \textit{best} reward by trying them. 
In most of the cases,
actions may affect not only the immediate reward but also the next state and consequently,
all subsequent rewards.

We can segment RL algorithms accounting for risk in two main categories: the first transform 
the optimization criterion to introduce the concept of risk, whereas the second, modifies
the exploration process to avoid exploratory actions that can lead to undesirable situations.\\
We will focus on the first category, which can be divided into 3 subcategories: worst-case
criterion, constrained criterion and risk-sensitive criterion.\\
Worst-case or minimax criterion has been studied by \citet{Heger1994}, 
\citet{Coraluppi1997} and \citet{Coraluppi1999},
in which the objective is to compute a control policy that maximizes the expected return
with respect to the worst case scenario. In general, minimax criterion has
been found to be too restrictive as it takes into
account severe but extremely rare events which may never occur.\\
Constrained criterion  aims to maximize the expected return while keeping
other types of expected utilities higher than some given bounds \citep{Altman1993}.
It may be seen as finding the best policy $\pi$ in the space of considered safe policies.
This space may be restricted by different constraints such as ensuring that
the expectation of costs \citep{Geibel2006} or the variance of return don't exceed
a given threshold \citep{Tamar2012}.
These constraint problems can be converted to equivalent unconstrained ones by using
penalty methods or a Lagrangian approach.\\
Finally, risk-sensitive criterion uses other utility metrics to be maximized instead of
expectation of cumulative rewards.
Lot of research has been done using exponential utility functions
\citep{Howard1972,Chung1987}.

Our approach aims to maximize
another risk metric, namely the Conditional Value at Risk (CVaR).
Non-formally, CVaR of a return distribution at confidence level $\alpha$ can be defined as the expected
cumulative reward of outcomes worse than the $\alpha$-tail of the distribution.
CVaR has recently gained a lot of popularity due to its mathematical properties \citep{Artzner1999},
which makes its computation and optimization much easier than for other metrics  \citep{Rockafellar2000}.
For example, 
it has recently been identified by \citet{Majumdar2020} as a suitable metric
for measuring risk in robotics.

\section{Related Work}

CVaR optimization aims to find the parameters $\theta$ that maximizes 
$\text{CVaR}_\alpha (Z)$, where the return distribution $Z$ is parameterized by a 
controllable parameter $\theta$ such that: $Z = f(X; \theta)$.In the simplest scenarios, 
where $X$ is not dependant on $\theta$, CVaR optimization may 
be solved using various approaches such as in \citet{Rockafellar2000}.\\
However, in RL, the tunable parameter $\theta$ also affects the 
distribution of $X$ and therefore, the standard existing approaches for CVaR optimization 
are not suitable.\\
Additionally, most of the work with such a goal has been done in the context of MDPs when
the model is known  by using dynamic programming methods \citep{Chow2015, Petrik2012}
but not much research has been done for the RL framework.

Under the RL framework, \citet{Morimura2010a} and \citet{Morimura2010b}
focused on estimating the density of the returns
to handle CVaR risk criteria. However, the resulting distributional-SARSA-with-CVaR algorithm
they propose has  proved  effectiveness  only  in  very simple  and discrete MDPs.

\citet{Tamar2015a} proposed a policy gradient (PG) algorithm for CVaR optimization, by
deriving a sampling based estimator for the gradient of CVaR and using it to optimize the
CVaR via stochastic gradient descent. However, they only considered continuous loss distributions
and they used empirical $\alpha$-quantile to estimate 
$\text{VaR}_\alpha$ which is known to be a biased estimator for small samples.\\
\citet{Chow2014} also proposed a PG algorithm for mean-CVaR optimization, which has several
disadvantages also shared with \citet{Tamar2015}. First, being PG algorithms,
they suffer from high variance on the gradient estimates, especially when the trajectories
are long. This high variance is even more exacerbated when using very low quantiles $\alpha$ for the CVaR 
since the averaging on the rewards is effectively only on $\alpha$N samples.
Second, they are both very sample inefficient since only the rewards below the quantile
are used to compute the gradients. Third, they are both trajectory-based (not incremental), 
i.e they only update the parameters after observing one or more full trajectories.\\
\citet{Chow2014} also proposed both a trajectory-based and incremental actor-critic
approach which help in reducing the variance of PG algorithms. However, they require
state augmentation of the original MDP formulation to a state-space
$\mathcal{X} \times \mathcal{S}$ where $\mathcal{S} \in \mathbb R$ is an additional continuous
state that allows to reformulate the Lagrangian objective function as an MDP. Despite this augmentation,
they still need to make use of several approximations to allow the algorithm be fully incremental and 
avoid sparse updates of the parameters. 

It is also important to be noticed that all the aforementioned algorithms are on-policy approaches.
This is first another source of sample inefficiency
because they cannot exploit data from experts or other sources. Additionally they
constraint the learning process to happen online while interacting with the environment,
which in real-case scenarios, it is paradoxically risky.


\section{Our approach}

In this thesis we present $\thename$, a CVaR optimization approach based
on a model-free, off-policy
deterministic actor-critic algorithm using deep function approximators. 
Instead of empirically estimating the VaR from the observed rewards or using the CVaR
Bellman equation, we build upon recent research in distributional RL 
\citep{Bellemare2017,Dabney2018a,Dabney2018b}
to estimate the full
\textit{value distribution} (i.e. the distribution of the random return received by a RL agent)
and we adapt it to the continuous setting using deterministic policy gradients.
A critic learns a parameterization of the value distribution and 
the actor is trained towards policies that maximize the CVaR of the returns 
by sampling from the parameterized value distribution and performing stochastic gradient ascent.

The CVaR optimization algorithm that we introduce has 2 remarkable properties, 
namely being an off-policy algorithm and using deterministic policies.\\
Being off-policy means that the algorithm can learn the optimal policy from data collected
from another policy. Hence, we can use a more exploratory \textit{behavior} policy to interact
with the environment and collect observations, and then use them to train
the optimal \textit{target} policy. Additionally to the fact of making exploration easier compared to on-policy algorithms,
being off-policy allows the possibility to learn from data generated by a conventional
non-learning controller or from a human expert, namely learning under, how is called in
literature, \textit{Batch RL} or \textit{offline RL} setting.
This application is really appealing when working with risky environments, when we do
not want to expose the learning agent to its risks at the earlier stages of the learning
process.\\
With this goal, we build upon $\thename$, and propose a second algorithm named $\thenameoff$, which
by introducing a action generative model, allows us to implement the CVaR optimization algorithm
in a fully off-line setting.


The second property is the fact that we use a target policy which is deterministic.
Using stochastic policies is not natural in many applications and it
could increase the variance of the return \citet{Taleghan2018}.
When again, working in an stochastic 
environment and with the goal of finding risk-aware policies, using a policy which is stochastic 
sounds quite counterintuitive.
Additionally, from a practical viewpoint, using stochastic policies requires integrating over both
state and action spaces to compute the policy gradient, whereas the deterministic case
only needs to integrate over the state space. Hence, stochastic policy gradients
may require much more samples, especially if the action space has many dimensions \citep{silver2014b}.

We would like to mention the recent research by \citet{BarthMaron2018} where they also adapt
the distributional perspective on RL to the continuous setting using deterministic policy gradients
and show state of the art performance in a set of difficult manipulation and locomotion tasks.
However, they use a more restrictive distribution representation and they do not focus on risk-sensitive
policies but still on maximizing the expected return.

With the same goal of CVaR optimization, we would like to mention the recent research by \citet{tang2020} in
which they also use deterministic policy gradients and a distributional critic. However,
their approach differs from ours since they decide to approximate the return distribution up
to its second order-statistics, i.e represent it using its mean and variance.
They model the distribution as a Gaussian and in
that way they can compute a closed-from calculation for the CVaR.
Despite they show its top performance in a simulated self-driving environment we consider
that using a Gaussian approximation for the return distribution can be a bit restrictive,
because it is light-tailed and unimodal. One of the motivations for using CVaR to assess
risk instead of other metrics is the fact that it accounts for
the distribution of the tails. Using a  return distribution which is unimodal may lead to omission of
bumps in the left-tail of the distribution which wastes one of the main 
properties of the CVaR as a metric.

We show performance of both $\thename$ and $\thenameoff$ algorithms.
The first in a setting in which the environment simulator is given and
the agent can interact with it during training to collect new data.
The second, and most importantly, in a setting in which the agent is trained completely
offline by using an external dataset and when no further interaction with the environment is allowed.

We first test performance in a toy car example and then we upgrade towards more complex environments
from D4RL \citep{d4rl}, a recent suite of tasks and datasets for benchmarking progress in offline RL.
Specially we focus on robotic tasks developed in the Mujoco Physics simulator \citep{Todorov2012},
but we are confident our approach could be extended to other areas, such as 
for medical applications or autonomous navigation.
We show our algorithm ability to minimize risk in terms of 
tail performance by introducing
stochasticity in the original cost function of the environments and comparing the policies
learnt by the risk-averse setting and by the risk-neutral one.

We hereby confirm $\thename$ and $\thenameoff$ algorithms are able to learn risk-sensitive policies in
complex high-dimensional, continuous action spaces.

