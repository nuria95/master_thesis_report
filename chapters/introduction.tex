\chapter{Introduction}
\label{sec:introduction}


% Decision-making is the process of taking choices by identifying an optimal strategy
% from current system states. 
In sequential decision-making problems, an agent interacts with an environment by 
selecting actions and in turn, it observes the state transitions of the system and it receives a reward.
The decision-maker or \textit{agent} uses a performance measure to assess how good actions were.
The most common optimization criterion is the expectation of the cumulative reward
collected by the agent. This leads to the so called \textit{risk-neutral} behavior.

When an agent interacts with the environment, different sources of irreducible 
randomness are introduced due to modelling errors or inherent stochasticity of the same.
Notion of risk in RL is related to the fact that even an optimal policy with respect to the 
expected return may perform poorly in some cases.
Since maximizing the expected return does not necessarily imply the avoidance
of rare occurrences of large negative outcomes,
we need other criteria to evaluate risk. \citep{Garcia2015}.

Risk-sensitive decision-making provides a promising
approach to compute robust and safe policies, essential in safety-critical applications,
such as autonomous navigation.
However, finding computationally tractable and conceptually meaningful methodologies for
such a goal is non-trivial and still a challenge.

We can segment RL algorithms accounting for risk in two main categories: the first transform 
the optimization criterion to introduce the concept of risk, whereas the second, modifies
the exploration process to avoid exploratory actions that can lead to undesirable situations.\\
We will focus on the first category, which can be divided into 3 subcategories: worst-case
criterion, constrained criterion and risk-sensitive criterion.\\
Worst-case or minimax criterion has been studied by \citet{Heger1994}, 
\citet{Coraluppi1997} and \citet{Coraluppi1999},
in which the objective is to compute a control policy that maximizes the expcted return
with respect to the worst case scenario. In general, minimax criterion has
been found to be too restrictive as it takes into
account severe but extremely rare events which may never occur.\\
Constrained criterion  aims to maximize the expected return while keeping
other types of expected utilities higher than some given bounds \citep{Altman1993}.
It may be seen as finding the best policy $\pi$ in the space of considered safe policies.
This space may be restricted by different constraints such as ensuring that
the expectation of costs \citep{Geibel2006} or the variance of return doesn't exceed
a given threshold \citep{Tamar2012}.
This constraint problems can be converted to equivalent unconstrained ones by using
penalty methods or a Lagrangian approach.

Finally, risk-sensitive criterion uses other utility metrics to be maximized instead of
expectation of cumulative rewards.
Lot of research has been done using exponential utility functions
\citep{Howard1972,Chung1987}.\\
Our approach aims to maximize
another risk metric, namely the Conditional Value at Risk (CVaR).
Non-formally, CVaR of a return distribution at confidence level $\alpha$ can be defined as the expected
cumulative reward of outcomes worse than the $\alpha$-tail of the distribution.
CVaR has recently gained a lot of popularity due to its mathematical properties \citep{Artzner1999},
which makes its computation and optimization much easier than for other metrics  \citep{Rockafellar2000}.
For example, 
it has recently been identified by \citet{Majumdar2020} as a suitable metric
for measuring risk in robotics.

CVaR optimization aims to find the parameters $\theta$ that maximizes 
$\text{CVaR}_\alpha (Z)$, where the return distribution $Z$ is parameterized by a 
controllable parameter $\theta$ such that: $Z = f(X; \theta)$.\\
In the simplest scenarios, where $X$ is not dependant on $\theta$ CVaR optimization may 
be solved using various approaches such as in \citet{Rockafellar2000}.\\
However, in RL, the tunable parameter $\theta$ also affects the 
distribution of $X$ and therefore, the standard existing approaches for CVaR optimization 
are not suitable.\\
Additionally, most of the work with such a goal has been done in the context of MDPs when
the model is known  by using dynamic programming methods \citep{Chow2015, Petrik2012},
and not much research has been done for the RL framework.

Under the RL framework, \citet{Morimura2010a} and \citet{Morimura2010b}
focused on estimating the density of the returns
to handle CVaR risk criteria. However, the resulting distributional-SARSA-with-CVaR algorithm
they propose has  proved  effectiveness  only  in  very  simple  and discrete MDPs.\\
\citet{Tamar2015a} proposed a policy gradient (PG) algorithm for CVaR optimization, by
deriving a sampling based estimator for the gradient of CVaR and using it to optimize the
CVaR via stochastic gradient descent. However, they only considered continuous loss distributions
and they used empirical $\alpha$-quantile to estimate 
$\text{VaR}_\alpha$ which is known to be a biased estimator for small samples.\\
\citet{Chow2014} also proposed a PG algorithm for mean-CVaR optimization, which has several
disadvantages also shared with \citet{Tamar2015}. First, by definition of PG algorithms,
they suffer from high variance on the gradient estimates, especially when the trajectories
are long. This high variance is even more exacerbated when using very low quantiles $\alpha$ for the CVaR 
since the averaging on the rewards is effectively only on $\alpha$N samples.
Second, they are both very sample inefficient since only the rewards below the quantile
are used to compute the gradient. Third, they are both trajectory-based (not incremental), 
i.e they only update after observing one or more full trajectories.\\
\citet{Chow2014} also proposed both a trajectory-based and incremental actor-critic
approaches which help in reducing the variance of PG algorithms. However, they require
state augmentation of the original MDP formulation to a state-space
$\mathcal{X} \times \mathcal{S}$ where $\mathcal{S} \in \mathbb R$ is an additional continuous
state that allows to reformulate the Lagrangian objective function as an MDP.

It is also important to be noticed that all the aforementioned algorithms are on-policy approaches.
This is first another source of sample inefficiency
because they cannot exploit data from experts or other sources. Additionally they
constraint the learning process to happen online while interacting with the environment,
which in real-case scenarios, it is paradoxically risky.

The CVaR optimization algorithm that we introduce has 2 main properties that so far,
we do not have knowledge any other algorithm for CVaR optimization in the RL setting has; 
namely, being an off-policy algorithm and using deterministic policies.\\
Being off-policy means that the algorithm can learn the optimal policy from data collected
from another policy. Hence, we can use a more exploratory \textit{behavior} policy to interact
with the environment and collect observations, and then use them to train
the optimal \textit{target} policy. 
Additionally to the fact of making exploration easier compared to on-policy algorithms,
being off-policy allows the possibility to learn from data generated by a conventional
non-learning controller or from a human expert, setting mostly called in literature as
\textit{Batch RL} or \textit{offline RL}.
This application is really appealing when working with risky environments, when we do
not want to expose the learning agent to its risks at the earlier stages of the learning
process.


The second property is the fact that we use a target policy which is deterministic.
Using stochastic policies is not natural in many applications and it
could increase the variance of the return \todo{\citet{Feinberg2008}/\citet{Taleghan2018}/nothing}.
When again, working in an stochastic 
environment and with the goal of finding risk-aware policies, using a policy which is stochastic 
sounds quite counterintuitive.

In this thesis, we propose a CVaR optimization approach based on a model-free, off-policy
deterministic actor-critic algorithm using deep function approximators. 
Instead of empirically estimating the VaR from the observed rewards or using the CVaR
Bellman equation, we build upon recent research in distributional RL 
\citep{Bellemare2017,Dabney2018a,Dabney2018b}
to estimate the full
\textit{value distribution} (i.e. the distribution of the random return received by a RL agent).
A critic learns a parameterization of the value distribution and 
the actor, is trained towards policies that maximize the CVaR of the returns 
by sampling from the parameterized value distribution and performing stochastic gradient ascent.

We show algorithm performance in two different settings.
First, in the most common off-policy RL setting in which the environment simulator is given and
the agent can interact with it to collect new data from time to time.
Second, and most powerful, the agent is trained completely offline by using an external
dataset and no further interaction with the environment is allowed.
For the 2 settings, the algorithm learns risk-sensitive policies in
complex high-dimensional, continuous action spaces
such as for some modified environments from the Open AI Gym toolkit and from D4RL,
a recent suite of tasks and datasets for benchmarking progress in offline RL.
