\section{Other distortion risk measures}

In this section we present other distortion risk measures $\phi$ a part from 
the one presented for CVaR.
The fact of being able to choose other mappings gives to the algorithm the flexibility of 
learning to optimize other risk measures rather than CVaR, to find other types of policies instead.
Algorithms that directly derive policies for CVaR optimization \citep{Chow2014},\citep{Tamar2015a} )
don't show this flexibility.

As discussed, evaluating under different distortion risk measures is equivalent to
changing the the distribution used to sample the quantile levels $\tau$.
For the CVaR:
\begin{equation}
    \text{CVaR}(\alpha, \tau)=\alpha\tau \quad \tau \sim U([0,1])
\end{equation}

A distortion risk measure proposed by \citet{Wang2000} can easily switch between
risk-averse (for $\eta<0$) and risk-sensitive (for $\eta>0$) distortions:

\begin{equation}
    \text{Wang}(\eta, \tau) = \Phi (\Phi ^{-1}(\tau) + \eta) \quad \tau \sim U([0,1])
\end{equation}

where $\Phi$ is the cumulative distribution of the standard Normal distribution.
While both Wang and CVaR heavily shift the distribution mass towards the
tails of the distribution,  CVaR entirely ignores all values corresponding to $\tau$>$\eta$
whereas Wang gives to these non-zero, but vanishingly small probability. \citep{Dabney2018b}.
Another, distortion risk measure would be a simple power formula
for risk-averse (for $\eta<0$) or risk-seeking(for $\eta>0$) policies:

\begin{equation}
    \text{Pow}(\eta,\tau)  = \left\{
	    \begin{array}{ll}
		 \tau ^ {\frac{1}{1+|\eta|}}      & \mathrm{if\ } \eta \ge 0 \\
		 1-(1-\tau)^{\frac{1}{1+|\eta|}}      & \mathrm{otherwise }
        \end{array}
        \right.
\end{equation}

\section{Truncation}

Changing the sampling distribution for the quantile levels to compute the CVaR of the
distribution makes use both the dual formulation and Acerbi's integral formula for CVaR.
A second approach would be to use the \citet{Rockafellar2000} optimization problem
for CVaR presented in \ref{eq:rockafellar} and as a reminder:
\begin{equation}
    \text{CVaR}_\alpha (Z) = \underset{\nu} \max \big\{\nu + \frac{1}{\alpha} \mathbb E_Z[[Z- \nu]^-]\big\} \label{eq:rockafellar_repeat}
\end{equation}

which in the optimal point it holds that $\nu^*=\text{VaR}_\alpha(Z)$.
Given the fact that our critic network maps to the respective quantile values of the target
return distribution, we can hence, compute the CVaR using \ref{eq:rockafellar_repeat}
by sampling uniformly from  the whole quantile distribution and then substracting the current
estimated VaR for the confidence level of our interest.
The deterministic risk-sensitive policy gradient would become:
\begin{align}
    \nabla_{\theta^\pi} J_\beta(\pi | \theta^\pi) &\approx \mathbb E_{x \sim \rho^\beta} 
\big [\nabla_{\theta^\pi} \pi(x,| \theta^\pi) \nabla_a  [\nu + \frac{1}{\alpha} \frac{1}{K}
\sum_{i=1}^K Z(x,a; \tau_i) | \theta^Z)]|_{a=\pi(x| \theta^\pi)}  \big] \nonumber
\end{align}

where $\tau_i \sim U([0,1])$ and $\nu = Z(x,a; \alpha \vert \theta^Z)$.
