\section{Other distortion risk measures}

In this section we present other interesting distortion risk measures $\phi$ that can be used
to find policies optimizing other metrics rather than the CVaR.
As discussed, evaluating under different distortion risk measures is equivalent to
changing the distribution used to sample the quantile levels $\tau$.
The fact of being able to switch the objective function just by changing the sampling distribution
for $\tau$ gives to the algorithm a lot of versatility.
Algorithms that directly derive policies for CVaR optimization \citep{Chow2014,Tamar2015a}
don't show such property.

For the CVaR we have the distribution:
\begin{equation}
    \text{CVaR}(\alpha, \tau)=\alpha\tau \quad \tau \sim U([0,1])
\end{equation}

A distortion risk measure proposed by \citet{Wang2000} can easily switch between
risk-averse (for $\eta<0$) and risk-sensitive (for $\eta>0$) distortions:
\begin{equation}
    \text{Wang}(\eta, \tau) = \Phi (\Phi ^{-1}(\tau) + \eta) \quad \tau \sim U([0,1])
\end{equation}

where $\Phi$ is the cumulative distribution of the standard Normal distribution.\\
While both Wang and CVaR heavily shift the distribution mass towards the
tails of the distribution,  CVaR entirely ignores all values corresponding to $\tau>\alpha$,
whereas Wang gives to these non-zero, but vanishingly small probability \citep{Dabney2018b}.\\
Another, distortion risk measure would be a simple power formula
for risk-averse (for $\eta<0$) or risk-seeking(for $\eta>0$) policies:
\begin{equation}
    \text{Pow}(\eta,\tau)  = \left\{
	    \begin{array}{ll}
		 \tau ^ {\frac{1}{1+|\eta|}}      & \mathrm{if\ } \eta \ge 0 \\
		 1-(1-\tau)^{\frac{1}{1+|\eta|}}      & \mathrm{otherwise }
        \end{array}
        \right.
    \quad \tau \sim U([0,1])
\end{equation}

\section{Truncation}

When changing the sampling distribution for $\tau$ to $U([0,\alpha])$
to compute the CVaR of the
distribution, we make use of both the dual formulation \ref{eq:dual_cvar} and Acerbi's
integral formulas for CVaR.\ref{eq:cvar_defs_repeat}.\\
A second approach would be to use the \citet{Rockafellar2000} optimization problem
for CVaR presented in \ref{eq:rockafellar} and as a reminder:
\begin{equation}
    \text{CVaR}_\alpha (Z) = \underset{\nu} \max \big\{\nu + \frac{1}{\alpha} \mathbb E_Z[[Z- \nu]^-]\big\} \label{eq:rockafellar_repeat}
\end{equation}

which in the optimal point it holds that $\nu^*=\text{VaR}_\alpha(Z)$.\\
Given the fact that our critic network estimates the $\text{VaR}_\alpha(Z)$ of the
return distribution, we can compute the CVaR using \ref{eq:rockafellar_repeat}
by sampling uniformly from  the whole quantile distribution, substracting the current
estimated VaR for the confidence level  $\alpha$ and truncating the result, that is to say,
we can compute CVaR via sampling by:
\begin{align}
    \text{CVaR}_\alpha(Z) \approx \nu + \frac{1}{\alpha} \frac{1}{K}
\sum_{i=1}^K \big[ Z(x,a; \tau_i\vert \theta^Z) - \nu \big]^- 
\end{align}

where $\tau_i \sim U([0,1])$ and $\nu = Z(x,a; \alpha \vert \theta^Z)$.

A disadvantage of such approach is sample-inefficiency, since due to truncation, 
only approximately $\alpha$K samples will be used to compute the policy gradient.

This is a shared disadvantage with the algorithm presented in
\citet{Chow2014}. For the sake of comparison we present it in the next section.
