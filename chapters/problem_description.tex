\chapter{Problem description and Related work}
\label{sec:problem_description}
\section{Problem description}

Standard reinforcement learning approaches aim to find policies which maximize the expected cumulative reward.
However, this approach neither takes into account variability of the reward around its mean
neither sensitivity of the policy to modelling errors.\\
In this thesis, we change the objective function of the
standard RL approach to one that optimizes another metric of the reward distribution which
takes into account the risk of the actions taken by the agent.
While several metrics have been designed to assess risk, we will focus on a particular risk metric
called Conditional Vale at Risk (CVaR).\\
We, thereby, present a RL algorithm that aims to find policies with optimal CVaR.

\subsection{Reinforcement Learning}

Reinforcement learning is an approach to learn a mapping from situations or states to actions so as to maximize 
a numerical reward signal. The learner or \textit{agent} is not told which actions to take
but instead must discover which actions yield the most reward by trying them. In most of the cases,
actions may affect not only the immediate reward but also the next state and consequently, all subsequent rewards.
These two characteristics—trial-and-error search and delayed reward—are the two most important
distinguishing features of reinforcement learning \cite{Sutton1998}.

\subsubsection{Markovian Decision Processes (MDPs)}
We formalize the problem of RL by using the framework of Markov decision processes (MDPs)
to define the interaction between a learning agent and its environment in terms of states,
actions and rewards.

MDPs  are discrete-time stochastic control processes which provide a
mathematical framework for modeling sequential decision making in situations where outcomes are
partly random and partly under the control of a decision maker and where actions influence not only immediate rewards,
but also future ones.\\
An MDP is defined by a tuple ($\mathcal{S,A},R,P,\gamma$), where $\mathcal{S,A}$ are the
state and action spaces respectively, $R(x,a)$ is the reward distribution, $P(\cdot |x,a)$ is
the transition probability distribution and $\gamma \in [0,1]$ is the discount factor. 

State transitions of an MDP satisfy the Markov property, in which
the set of transition probabilities to next states depend only on the current state and action of the system,
but are conditionally independent of all previous states and actions. Hence, the state must
provide information about all aspects of the past agent-environment interaction that
make a difference for the future.

Solving an MDP involves determining a sequence of policies $\pi$ (mapping states to actions)
that maximize an objective function.
A commonly considered class of policies in literature are the class of Markovian policies $\Pi_M$ where
at each time-step \textit{t} the policy $\pi_t$ is a function that maps state $x_t$ to 
the probability distribution over the action space $\mathcal{A}$. 
In the special case when the policies are time-homogeneous, i.e. $\pi_j = \pi \; \forall j \geq 0$ then the
class of policies is known as stationary Markovian $\Pi_{M,S}$.\\
This set of policies, under which actions only depend on current state information and
its state-action mapping is time-independent, makes the problem of solving for an optimal policy more
tractable and common solution techniques involve dynamic programming algorithms \citep{Bertsekas1995}
such as Bellman iteration.\\
When the objective function is given by a risk-neutral expectation of cumulative reward,
i.e.:
\begin{equation}
    \underset{\pi \in \Pi_H}{\min} \mathbb E  \big [  \sum_{t=0}^{\infty} \gamma^t R(x_t,a_t) | x_0, a_t \sim \pi_t(\cdot |h_t)  \big]
\end{equation} 

where $ \Pi_H$ represents the set of all history-dependant policies,
the Bellman's principle of optimality \citep{Bertsekas1995} shows that the optimal
policy lies in the class of stationary Markovian policies $\Pi_{M,S}$

When dealing with other types of objective functions that aim towards more risk-sensitive
policies, these nice properties doesn't normally hold
and require extra mathematical formulations, such as MDP state augmentation \citep{Chow2015}.



\section{Risk}
Standard reinforcement learning agents aim to maximize the expected cumulative reward and hence
do not take risk into account. In some scenarios the shape of the reward distribution might be unimportant,
since highly different distributions still can have same expectation. However, in real world scenarios, 
in  when catastrophic losses can occur, risk must be taken into account.\\
We can find 3 types of strategies with respect to risk, namely \textit{risk neutral, risk averse} and \textit{risk seeking}

As an example, suppose a participant in a game is told to choose between two doors.
One door hides 1000CHF and the other 0CHF. The host also allows the contestant to 
take 500CHF instead of choosing a door. The two options (choosing between door 1 and door 2, or taking 500CHF)
have the same expected value of 500CHF. But it can clearly be seen that the risk among two options is different.
Since the expected value is the same, risk neutral contestant is indifferent between these choices.
A risk-seeking contestant will maximize its utility from the uncertainty and hence choose a door,
whereas the risk-averse contestant  will accept the guaranteed 500CHF.

We can segment RL algorithms accounting for risk in two main categories: the first transform 
the optimization criterion to introduce the concept of risk, whereas the second, modifies the exploration
process to avoid exploratory actions that can lead to undesirable situations.\\
We will focus on the first category, which can be divided into 3 subcategories: worst-case criterion,
constrained criterion and risk-sensitive criterion.\\
Worst-case or minimax criterion has been studied by \citet{Heger1994}, \citet{Coraluppi1997} and \citet{Coraluppi1999},
in which the objective is to compute a control policy that maximizes the expectation of the return
with respect to the worst case scenario. In general, minimax criterion has been found to be too restrictive as it takes into
account severe but extremely rare events which may never occur.\\
Constrained criterion  aims to maximize the expectation of the return while keeping other types of expected
utilities higher than some given bounds \citep{Altman1993}. It may be seen as finding the best policy $\pi$
in the space of considered safe policies. This space may be restricted by different constraints: ensuring that
the expectation of return exceeds some specific minimum threshold \citep{Geibel2006}
or that te variance of return doesn't exceed a given threshold \citep{Tamar2012}. This constraint problems
can be converted to equivalent unconstrained ones by using penalty methods or a Lagrangian approach.\\
Finally, risk-sensitive criterion use other utility metrics to be maximized instead of expectation of cumulative rewards.
Lot of research has been done using exponential utility functions \citep{Howard1972,Chung1987}. \\
A risk metric that has recently gained a lot of popularity is the Conditional Value at Risk,
due to its favorable computation properties and superior ability to safeguard a
decision maker from the "outcomes that hurt the most" \cite{Serraino2013}.\\
In this thesis we present an optimization approach for this metric.

\subsection{Conditional Value-at-Risk (CVaR)}

Let Z be a bounded-mean random variable, i.e $\mathbb E[|Z|] < \infty$, on a probability space 
$(\Omega, \mathbb F, \mathbb P)$, with cumulative distribution
function $F(z) = \mathbb P (Z \leq z)$. We interpret Z as the reward distribution.
The value-at-risk (VaR) at confidence level $\alpha \in (0,1) $ is the $\alpha$-quantile of Z, i.e, 
$\text{VaR}_\alpha (Z) = \inf \{ z \hspace{1mm} | \hspace{1mm}F(z) \geq  \alpha  \}$.
The conditional value-at-risk (CVaR) at confidence level $\alpha \in (0,1) $ is defined as
the expected reward of outcomes worse than the $\alpha$-quantile ($\text{VaR}_\alpha$):
\begin{equation}
    \text{CVaR}_\alpha (Z) = \mathbb E [Z | Z \leq \text{VaR}_\alpha]= \frac{1}{\alpha} \int_{0}^{\alpha} F^{-1}_Z(\beta) d\beta=\frac{1}{\alpha} \int_{0}^{\alpha} \text{VaR}_\beta (Z) d\beta
 \end{equation}
While both VaR and CVaR are risk measures, only CVaR is coherent in the sense of \citet{Artzner1999}.
In addition, CVaR takes into account the possibility of tail events where losses exceeds VaR whereas VaR
is incapable of distinguishing situations beyond it.

Given its properties, \citet{Rockafellar2000} also showed that CVaR is equivalent to the solution of
the following optimization problem:

\begin{equation}
    \text{CVaR}_\alpha (Z) = \underset{\nu} \max \big\{\nu + \frac{1}{\alpha} \mathbb E_Z[[Z- \nu]^-]\big\}
\end{equation}

where $(x)^- = \min(x,0)$.\\
In the optimal point it holds that $\nu^*=\text{VaR}_\alpha(Z)$.

A useful property of CVaR, is its alternative dual representation \cite{Artzner1999}:

\begin{equation}
    \text{CVaR}_\alpha (Z) = \underset{\xi \in U_{\text{CVaR}} (\alpha, \mathbb{P})} \min \mathbb E_\xi[Z]
\end{equation}

where $\mathbb E_\xi[Z]$ denotes the $\xi$-weighted expectation of $Z$, and the risk envelope $U_\text{CVaR}$ is
given by:

\begin{equation}
    U_{\text{CVaR}}(\alpha, \mathbb{P}) = \Big\{\xi | \xi(w)  \in \big [ 0, \frac{1}{\alpha} \int_{w\in\Omega} \xi(w)\mathbb{P}(w)dw=1   \big ] \Big\}
\end{equation}

Thus, the CVaR of a random variable may be interpreted as the worst case expectation of $Z$, under
a perturbed distribution $\xi \mathbb{P}$.

Due to its superior mathematical properties and practical implications, CVaR optimization has gained a lot interest in the
risk-sensitive literature.\\
CVaR optimization aims to find the parameters $\theta$ that maximizes $\text{CVaR}_\alpha (Z)$, where the
reward distribution $Z$ is parameterized by a controllable parameter $\theta$, such that: $Z = f(X; \theta)$.\\
In the simplest scenarios, where $X$ is not dependant on $\theta$ CVaR optimization may be solved using
various approaches such in \citet{Rockafellar2000}.\\
However, in many domains, such as reinforcement learning, the tunable parameter $\theta$ also affects the 
distribution of $X$, and hence the standard existing approaches for CVaR optimization are not suitable.\\
Additionally, most of the work in these domains, has been done in the context of MDPs when
the model is known \citep{Chow2015, Petrik2012}, by using dynamic programming methods
and not much research has been done for the RL framework.

\citet{Morimura2010a} and \citet{Morimura2010b}
focused on estimating the density of the returns
to handle CVaR risk criteria. However the resulting distributional-SARSA-with-CVaR algorithm they propose
has  proved  effectiveness  only  in  very  simple  and discrete MDP.\\
\citet{Tamar2015a} proposed a policy gradient (PG) algorithm for CVaR optimization, by deriving a 
sampling based estimator for the gradient of CVaR and used it to optimize the CVaR by stochastic gradient descent.
However they only considered continuous loss distributions and they used empirical $\alpha$-quantile to estimate 
$\text{VaR}_\alpha$ which is known to be a biased estimator for small samples.\\
\citet{Chow2014} also proposed a PG algorithm for mean-CVaR optimization, which has several disadvantages
also shared with \citet{Tamar2015}. First, by definition of PG algorithms, they suffer from
high variance on the gradient estimates, especially when the trajectories are long. This high variance
is even more exacerbated when using very low quantiles $\alpha$ for the CVaR 
since the averaging on the rewards is effectively only on $\alpha$N samples.
Second, they are both very sample inefficient since only the rewards below the quantile
are used to compute the gradient.\\
Third, they are both trajectory-based (not incremental), 
i.e they only update after observing one or more full trajectories.\\
\citet{Chow2014} also proposed both a trajectory-based and incremental actor-critic approaches which help
in reducing the variance of PG algorithms. However, they require state augmentation of the original MDP formulation to a
state-space $\mathcal{X} \times \mathcal{Y}$ where 
$\mathcal{Y} = (0,1]$ is an additional continuous state that represents the different confidence levels. 
This allows to include as a critic the CVaR value function and it is necessary due to 
the time-inconsistency of CVaR metric.
It is also important to be noticed, that all the aforementioned algorithms are on-policy approahces.
This is first another source of sample inefficiency
because they cannot exploit data from experts or other sources. Additionally they constraint the 
learning process to happen online while interacting with the environment, which in real-case scenarios,
for example robotics, can lead to very detrimental situations for the robot and hence it is,
paradoxically, risky.

Most recently, \citet{Dabney2018b} presented a distributional RL approach to train risk-averse and risk-seeking agents,
using, among others, CVaR as risk objective. The approach uses Q-learning
and hence, it is only suitable for discrete action spaces.

We work towards sample efficient RL algorithms that can learn optimal CVaR policies.
no biased alpha quantile since it is estimated by the Z distribution not sampled
off--policy
deterministic
continuous actions
incremental


In this thesis, we present an off-policy approach for CVaR optimization using actor-critic approach based on the DPG algorithm.
Instead of estimating the VaR empirically from the observed rewards or using CVaR Bellman equations,
we rely on the recent advances on distributional RL  \citep{Bellemare2017,Dabney2018a,Dabney2018b} to estimate the full
\textit{value distribution} (i.e. the distribution of the random return received by a RL agent) and from there, compute the
\todo{too ambiguous/incorrect?} \textit{estimated empirical} CVaR.






